%% \documentclass[a4paper,10pt,oneside,onecolumn]{article}
%% \usepackage[utf8]{inputenc}
%% % \usepackage{times}
%% \usepackage{a4wide}
%% \usepackage[T1]{fontenc}
%% % \usepackage[pdftex,a4paper]{geometry}
%%\usepackage[pdftex,a4paper, total={15.4cm,23cm}, top=3.8cm, left=2.8cm, includefoot]{geometry}
%% \usepackage{array}
%% \usepackage{times,colordvi,epsfig,float, colortbl,multicol, xcolor}
%% \usepackage{graphics}
%% \usepackage{graphicx}
%% \usepackage{framed}
%% % \usepackage[dvips]{graphicx}
%% \usepackage{amsmath}
%% \usepackage{url}
%% \usepackage{comment}
%% \usepackage{picinpar}
%% \usepackage{wrapfig}
%% \usepackage{epsfig}
%% \usepackage{multirow}
%% \usepackage{fancyhdr}
%% \usepackage{lettrine}
%% \usepackage{natbib}
%% \usepackage{multicol}
%% \pagestyle{fancy}
% -----


%% \usepackage{color}
%%   \usepackage[colorlinks,hyperindex,plainpages=false,pdftex, linkcolor=black,citecolor=black,urlcolor=black,pagecolor=black]{hyperref}
%%   \def\pdfBorderAttrs{/Border [0 0 0] } % bez okrajù kolem odkazù
%%   \pdfcompresslevel=9

%% \newcommand{\degree}{\ensuremath{^\circ}}
%% \newcommand\uv[1]{\quotedblbase \,#1\textquotedblleft}
%% % Title Page
%% \title{}
%% \author{Jaroslav Vážný}

%% % \usepackage[dvips]{hyperref}

%% \date{2011-09-22}

%% \renewcommand{\headrulewidth}{0.5pt} % tloušťka linky

%% % \fancyhf{} % smaže aktuální nastavení záhlaví a zápatí
%% % \fancyhead[LE,RO]{\bf } 
%% % \fancyhead[LO]{\bfseries\rightmark}
%% % \fancyhead[RE]{\bfseries\leftmark}
%% \lhead{Jaroslav Vážný}
%% \rhead{{{Brno PhD talent}}}
%% \renewcommand{\headrulewidth}{0.7pt} % tloušťka linky
%% \renewcommand{\footrulewidth}{0pt}   % patička chybí
%% % \addtolength{\headheight}{1.2pt} % prostor pro záhlaví

%% \makeatletter % @ se stane aktivnim znakem
%% \renewcommand\thesection {\thechapter\@Roman\c@section}
%% \makeatother  % @ se bude chovat jako obvykle


% \textwidth 15cm
% \oddsidemargin 1.8 cm
% \hoffset 0.1cm
% \voffset 0.3cm

% \textwidth 20cm
% \hoffset -2.35cm

%% \begin{document}

% \huge{\textsc{Scientific project}}\\
\vspace{2cm}
\begin{center}
\huge{\bf Virtual Observatory \& Data Mining:\\ Astronomy for 21 century}\\
% the role of gas in the evolution\\ of minor merger system
\medskip
\large{Applicant: {Jaroslav Vážný}}\\
\large{Supervisor: RNDr. Petr Škoda, CSc.}
\end{center}
\noindent\hrulefill

\noindent \textbf{Key words:} Astroinformatics; Virtual Observatory; Data Mining;
DAME; Machine Learning; Decision Trees; Neural Networks; Support Vector
Machines

\section{Objectives and original contribution of the project}

%\paragraph{Background}
\bigskip
\renewcommand{\LettrineFontHook}{\color{red}}


\lettrine[lines = 3, loversize=-0.1, lraise=0.1]{F}{}rom the dawn of
its existence astronomy has been starving for data but in the last few
decades the situation has changed and now we are facing data deluge of
biblical proportions. The data are not just increasing in size but
also in complexity and dimensionality
\cite{ballastroinformatics}. Astroinformatics is the new field of
science which has emerged from this technology driven progress.
Virtual Observatory, Machine Learning, Data Mining, Grid Computing are
just few examples of new tools available to scientists.  
The astronomers are not alone and particle physics, biology and other
sciences are also in the vanguard of the data intensive science. This
is great opportunity for interdisciplinary collaboration.


%% \begin{wrapfigure}{r}{0.3\textwidth}
%%   \vspace{-19.5pt}
%%   \begin{center}
%%     \includegraphics[width=4cm]{astroinformatics}\label{40}
%%   \end{center}
%%   \vspace{-12pt}
%%   \vspace{-6pt}
%% \end{wrapfigure} 

The general goal of the project is to demonstrate the power of new
concepts and technologies of computers science in the context of
astrophysics. This project is a part of my long-term effort. During my
bachelor work I developed a process for discovering new candidates of
Blazars (super-massive black hole at the center of an galaxy) and in my
master thesis I utilized technologies of Virtual Observatory and Data
Mining to find new Be stars (rare, extremely rapid rotating stars). My
reviewers were prof. Giuseppe Longo and Dr. Massimo Brescia. They are
main figures behind the DAME project, one of the most important Data
Mining projects in current astrophysics, and they invited us to work
on common project. Since my supervisor has two more computer science
students we are now team with expertise in different ares. This
multidisciplinary international collaboration will be the basis for my
work. The first goal will be the implementation of the Decision Tree
algorithm in the DAME framework and its utilization in astrophysical
research.


\paragraph{The project}
Main “vision” is to create a joined astrophysical data mining project
based on the DAMEWARE \footnote{DAME Web Application REsource}
infrastructure and other DAME\footnote{DAta Mining \& Exploration:
  general purpose, Web-based, distributed data mining infrastructure
  specialized in Massive Data Sets exploration with machine learning
  methods } resources.  The common work  has two distinct
aspects to be carried out:

\begin{enumerate}
\item{Technological:} 

the goal is the design and development of a supervised machine learning model,
conceptually based on a classification tree (for instance a decision tree used in the data mining context),
able to solve multi-class classification problems. The model must be scalable, i.e. having a robust efficiency
to treat massive datasets, and with a sufficiently general behavior, i.e. able to digest both categorical and
numerical data. An important aspect of the work is that such model must be fully integrated into the
DAMEWARE and accessible by external users exclusively through its GUI (Graphical User Interface);

\item{Scientific:} 

the goal is to exploit the DAMEWARE and other resources (either
available or new), of DAME Program to perform astrophysical data
mining experiments with public data coming out from Virtual
Observatory, large programs or other warehouses.
\end{enumerate}







%% Overall contribution and goals of the project:
%% \begin{itemize}
%%  \item \vspace{-0.3cm} Set up Czech-Italian team
%%  \item \vspace{-0.3cm} Implement Decision Tree algoritm into DAME
%%  \item \vspace{-0.3cm} Use DAME on astrophysical phenomenas

%% \end{itemize}


\newpage
\section{Theoretical framework, applied methods and techniques,\\ basic references}
\paragraph{Data avalanche: Opportunity or disaster?}

\bigskip


There are two important trends in the current astronomical surveys:

\begin{itemize}

  \item{Size:} The cumulative compressed data holdings of the ESO archive will
    reach 1~Petabyte by 2012 \cite{hanisch2010international}. Projects
    like Large Synoptic Survey Telescope (LSST) will produce about 30
    TB per night, leading to a total database over the ten years of
    operations of 60 PB for the raw data \cite{becla2006designing}.
   
  \item{Complexity:} Modern surveys will cover the sky in different
    wave-bands, from gamma and X-rays, optical, infrared to radio. The
    ability to cross correlate these observations together may lead to
    new understanding of physical
    phenomenas \cite{hanisch2010international}.
\end{itemize}



\noindent Such an amount of data is not possible to transfer over the
network. Data resources are heterogeneous, distributed and
decentralized in their nature.

\paragraph{Virtual Observatory (VO)}

   \begin{wrapfigure}{r}{0.4\textwidth}
     \vspace{0pt}
     \begin{center}
       \ifpdf
       \includegraphics[width=0.4\textwidth]{ivoamembers}
       \else
       \includegraphics[bb = 92 86 545 742, height=6in]{ivoamembers.jpg}
       \fi
     \end{center}
     \vspace{-15pt}
%     \caption{IVOA members}
     \vspace{-5pt}
   \end{wrapfigure}


   For handling of heterogeneous distributed data it is necessary to
   have the set of common standards and protocols as well as an
   authority encouraging their implementation. Such an authority is
   the International Virtual Observatory Alliance (IVOA). It currently
   comprises 19~VO programs from Argentina, Armenia, Australia,
   Brazil, Canada, China, Europe, France, Germany, Hungary, India,
   Italy, Japan, Russia, Spain, the United Kingdom, and the United
   States and inter-governmental organizations (ESA and ESO)
   \cite{hanisch2010international}. Standards and specifications
   produced by IVOA can be obtained at \url{http://www.ivoa.net/}.


   \begin{wrapfigure}{r}{0.4\textwidth}
     \vspace{0pt}
     \begin{center}
       \ifpdf
       \includegraphics[width=0.4\textwidth]{architecture}
       \else
       \includegraphics[bb = 92 86 545 742, height=6in]{ivoamembers.jpg}
       \fi
     \end{center}
     \vspace{-15pt}
%        \caption{VO Architecture}
        \label{FigArchitecture}
     \vspace{-5pt}
   \end{wrapfigure}


The Architecture is depicted on the figure. The level of abstraction
goes from top to bottom. Starting with interfaces, used by people or
applications to discover resources.  The next level is the service
layer implemented by standard protocols, followed by the hardware
level where actual data are stored. This onion-like structure hides
the complexity of the lower layer and provide data and meta-data to
the higher layer. This concept is similar to TCP/IP \footnote{TCP/IP
  (Transmission Control Protocol/Internet Protocol).  The basic
  communication language or protocol of the Internet.}  protocol.



%\clearpage

    The VO architecture is serviced oriented. Each service is
    autonomous with well defined boundaries. The very important aspect
    of VO implementation is the adoption of formats and protocols used
    in astronomy (FITS\footnote{Flexible Image Transport System is
      the standard astronomical data format endorsed by both NASA and
      the IAU }) and Computer Science
    (XML\,\footnote{Extensible Markup Language (XML) is a set of rules
      for encoding documents in machine-readable form.} , Web
    Service\,\footnote{method of communication between two electronic
      devices over a network.}  SOAP\,\footnote{Simple Object Access
      Protocol, is a protocol specification for exchanging structured
      information in the implementation of Web Services in computer
      networks.}, REST\footnote{Representational State Transfer (REST)
      is a style of software architecture for distributed hyper-media
      systems such as the World Wide Web. Used in
      \url{www.youtube.com} }) for many years. In other words, VO does
    not try to reinvent the wheel but it stands on the shoulders of
    giants.


\paragraph{Data Mining}
Data Mining and related techniques are created exactly for such
purposes. Used correctly, it can be powerful approach, promising
scientific advance. On the other hand this field is very complex with
dozens of different methods and algorithms. This form needs and
opportunity for interdisciplinary cooperation with Data Mining
experts. This can be very beneficial for both fields, providing
astronomers with interesting methods for data analysis and computer
scientist with the large amount of quality data.
\paragraph{Supervised Methods}
These methods are also known as predictive\cite{ball2010data}. They
rely on training set with known target property. This set must be
representative. The selected method is trained on that set and the
result is then used on data for which the target property is not
known. Among supervised method are classification, regression, anomaly
detection and others.

\paragraph{Decision Tree (DT)} is an example of supervised
classification. Based on final number of data
$(x^{(1)}, x^{(p)})$ with known class $C_1,\ldots, C_m$
classifier is created, i.e. mapping $f$ classifying any $x \in
\mathcal{X}, f:\mathcal{X}\rightarrow \mathcal{Y}$, where
$\mathcal{X}$ is a set of possible input vectors and $\mathcal{Y}$ is
a set which values represent classes $C_1,\ldots, C_m$ (for example
$\mathcal{Y} = {1,\ldots,m}$). The model is constructed based on
training set as a tree structure, where leaves represent
classifications and branches conjunctions of features that lead to
those classifications.

%    \clearpage

    \paragraph{DAME}

    %% \begin{wrapfigure}{hlt}{0.4\textwidth}
    %%  \vspace{0pt}
    %%  \begin{center}
    %%    \ifpdf
%       \includegraphics[width=0.4\textwidth]{dame}
    %%    \else
    %%    \includegraphics[bb = 92 86 545 742, height=6in]{ivoamembers.jpg}
    %%   \fi
    %%  \end{center}
    %% \end{wrapfigure}

    (DAta Mining \& Exploration) is an innovative, general purpose,
    Web-based, distributed data mining infrastructure specialized in
    Massive Data Sets exploration with machine learning
    methods. Initially fine tuned to deal with astronomical data only,
    DAME has evolved in a general purpose platform program, hosting a
    cloud of applications and services useful also in other domains of
    human endeavor. DAME is an evolving platform and new services as
    well as additional features are continuously added. The modular
    architecture of DAME can also be exploited to build applications,
    finely tuned to specific needs. 

    The DAME web application is a joint effort between the
    Astroinformatics groups at University Federico~II, the Italian
    National Institute of Astrophysics, and the California Institute
    of Technology.  DAME aims at solving in a practical way some of
    the DM problems, by offering a completely transparent
    architecture, a user-friendly interface, and the possibility to
    seamlessly access a distributed computing infrastructure. DAME
    adopts VO standards in order to ensure the long-term
    interoperability of data; however, at the moment, it is not yet
    fully VO compliant. This is partly due to the fact that new
    standards need to be defined for data analysis, DM methods and
    algorithms development. In practice, this implies a definition of
    standards in terms of ontology and a well-defined taxonomy of
    functionalities to be applied in the astrophysical use
    cases \cite{brescia2011extracting}.  

\break

%\url{http://dame.dsf.unina.it/}


%\subsection*{References}
\bibliographystyle{plain}
\bibliography{references}



\section{Time schedule and key milestones of the project}
\begin{tabular}{p{2.3cm}p{12.6cm}} \vspace{-0.2cm} \bf Academic Year & \vspace{-0.2cm} \bf Milestones \\
{\bf{2011/2012}} & \multirow{2}{*}{}Set up international
(Czech-Italian) interdisciplinary (astrophysicists, computer
scientists) team.\\ & Study DAME Architecture.
 \end{tabular}
\begin{tabular}{p{2.3cm}p{12.6cm}} \vspace{-0.2cm} {\bf{2012/2013}}
  & \vspace{-0.2cm} Implement Decision Tree into DAME.\\ & Testing on
  various scientific cases (Study of AGN, Galaxies, Be stars). \\ &
  Choosing main scientific goal. \\
 \end{tabular}
 \begin{tabular}{p{2.3cm}p{12.6cm}} \vspace{-0.2cm} {\bf{2013/2014}}
   & \vspace{-0.2cm} Detailed analysis of chosen scientific goal.\\
 \end{tabular}
 \begin{tabular}{p{2.3cm}p{12.6cm}} \vspace{-0.2cm} {\bf{2014/2015}} & \multirow{2}{*}{}Publish papers to international conferences and high impact factor international journals.\\
 & Write a PhD dissertation using the project results and related topics. \vspace{-0.2cm} 
 \end{tabular}\\
\smallskip 


\noindent\textbf{During the all project stages:} The applicant will
participate in summer schools, conferences and workshops. Results
obtained during the project solving will be published throughout all
stages.



\bigskip
%\newpage 
\section{Institutions where the project will be implemented}

\begin{itemize} \addtolength{\itemsep}{-0.5\baselineskip}

\item {\bf{The Institute of Theoretical Physics and Astrophysics}}, Masaryk
University in Brno 

Home institution of the PhD study

\item {\bf{Astronomical Institute of the Academy of Sciences of the Czech
    Republic}}

Home institution of supervisor

\item {\bf{The Faculty of Information Technology (FIT) at Brno University of Technology}}

Home institution of other PhD and bachelor students from the team

\item {\bf{Department of Physical Sciences University of Napoli - Federico II}}

Home institution of DAME project

\end{itemize}

%% \footnote{The applicant has succeeded, in June 2011, in
%%   a competition for a part-time (25\%) PhD position at the
%%   Astronomical Institute ASCR.}

%% Team members:

%% \begin{multicols}{2}
%% \begin{itemize} \addtolength{\itemsep}{-0.5\baselineskip}
%% \item Massimo Brescia
%% \item Pavla Bromová
%% \item Giuseppe Longo
%% \item Vojtěch Rylko
%% \item Petr Škoda
%% \item Jaroslav Vážný
%% \end{itemize}
%% \end{multicols}


\break
\section{Expert consultants and their contribution to the project}

\textbf{RNDr. Petr Škoda, CSc.}, supervisor 
\begin{itemize}
\item Staff researcher at Astronomical Institute of the Academy of Sciences
\end{itemize}

%% FIT VUT\footnote{The Faculty of Information
%%   Technology (FIT) at Brno University of Technology}.

\bigskip
\noindent \textbf{prof. Giuseppe Longo}, DAME Principal Investigator
(PI), consultant
\setlength{\columnsep}{-15pt}
\begin{multicols}{2}
\begin{itemize} \addtolength{\itemsep}{-0.5\baselineskip}
\item University Federico II
\item Department of Physical Sciences
\item Faculty of Sciences - University Federico II
\item INAF - National Institute of Astrophysics
\item INAF - Napoli
\item INFN - National Institute of Nuclear Physics
\item INFN - Napoli
\item RIAA - National network Astr. \& Astrophysics
\item Accademia Pontaniana
\item International Astronomical Union
\item Meta Institute for Computational Astrophysics
\item California Institute of Technology
\end{itemize}
\end{multicols}
% \subsubsection*{Colaborates and consultants:}
\bigskip

\noindent \textbf{Dr. Massimo Brescia}, DAME Project Manager (PM),
consultant

\begin{itemize}\addtolength{\itemsep}{-0.5\baselineskip}
\item Astronomer researcher at INAF OAC Napoli, since 2000
\item Teacher of Astronomical Technologies, Faculty of Astrophysics \& Space Science, University Federico II of Napoli, since 2007
\item member of IAU, International Astronomical Union, since 2006
\item Scientific Association with Dep. of Physics Sciences, University Federico II of Napoli, since 2007
\end{itemize}

% -- Nemam zde vyjmenovat i cleny pidiskupiny, ze jde o spolupraci v ramci pidi?

\section{Relation between the applicant's project and doctoral thesis}
The project is tightly connected with and significantly extends the
topic of PhD thesis: \textit{Application of modern computer science in
  astronomy}. Results will available to the public through DAME
interface and could be used outside astrophysics field.


\section{Motivation for solving the project}
The searching for effective collaboration between natural and computer
science is one of the most important task of today's science. We are
witnessing the genesis of multiple x-informatics (astro, bio, etc)
sciences. This multidisciplinary cooperation is vital not only for
scientists but also for the public. The born of World Wide
Web\footnote{Developed as documentation tool during LEP project at
  CERN} is nice example. And at the end, there is no such things like
geology, chemistry or physics. There is just nature.





%\end{document}
